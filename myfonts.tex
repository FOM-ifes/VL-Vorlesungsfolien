%
% beamerfonttheme.tex / myfonts.tex
%
% - - - - - - - - - - - - - - - - - - - - - - - - - - - - - - - - - - - - - %
\ifxetex% XeLaTeX Fontsetting                                               %
% . . . . . . . . . . . . . . . . . . . . . . . . . . . . . . . . . . . . . .
\usepackage{xltxtra}
% Now done via _output.yaml ! 
% - - - - - - - - - - - - - - - - - - - - - - - - - - - - - - - - - - - - - %
\else%    pdf-LaTeX Fontsetting                                             % 
% . . . . . . . . . . . . . . . . . . . . . . . . . . . . . . . . . . . . . .
% # Latin Modern
% Die LaTeX Standardschrift ist die Latin Modern (lmodern Paket).
% If Latin Modern is not available for your distribution you must install the
% package cm-super instead. Otherwise your fonts will look horrible in the PDF
%
  \usepackage{lmodern}
% . . . . . . . . . . . . . . . . . . . . . . . . . . . . . . . . . . . . . .
%
% needed for all T1 fonts!
  \usepackage[T1]{fontenc} 	% sonst wird OT1 coding ausgegeben 
%							% und dann funktioniert Myriad (pmy) nicht mehr
  \usepackage{textcomp} 		% sonst wird OT1 coding ausgegeben 
%							% und dann funktioniert Myriad (pmy) nicht mehr

% . . . . . . . . . . . . . . . . . . . . . . . . . . . . . . . . . . . . . .
% # Palantino: Serifen für die Mathematischen Formeln (pplx, pplj)
\ifxetex\relax\else
\usepackage[%
  osf,
  sc
]{mathpazo} 
\fi
%\renewcommand{\sfdefault}{pplx}
%\renewcommand{\sfdefault}{pplj}
%
% . . . . . . . . . . . . . . . . . . . . . . . . . . . . . . . . . . . . . .
% # Schrifttype eulervm / Euler Virtual Math
%
% URL: 
% http://ftp.riken.jp/tex-archive/fonts/eulervm/doc/latex/eulervm/eulervm.pdf
%  Zitat: Do not use the Euler math fonts in conjunction with the default 
%         Computer Modern text fonts – this is ugly!
  \usepackage[                           %% --- EulerVM (MATH)
    small,       %for smaller Fonts
    OT1,
%   euler-digits % digits in euler fonts style
  ]{eulervm}
% 
% . . . . . . . . . . . . . . . . . . . . . . . . . . . . . . . . . . . . . .
% \DeclareSymbolFont{operators}   {OT1}{cmr} {m}{n}
  \DeclareSymbolFont{letters}     {OML}{zplm} {m}{it}
% \DeclareSymbolFont{symbols}     {OMS}{cmsy}{m}{n}
% . . . . . . . . . . . . . . . . . . . . . . . . . . . . . . . . . . . . . .
% ---------------------------------------------------------------------------
% eurosym für das Euro-Symbol
% =======
%
% Als Option "official" für das "offizelle" Eurosymbol
%
%\usepackage[official]{eurosym}
%
\usepackage{eurosym}
% Eurosymbol auch in mathematischen Formeln richtig darstellen
%
\DeclareRobustCommand{\officialeuro}{%
  \ifmmode\expandafter\text\fi
  {\fontencoding{U}\fontfamily{eurosym}\selectfont e}}
%
% \euro in UTF-8 als Standard-Eurozeichen definieren
%
  \DeclareUnicodeCharacter{20AC}{\euro}

%
\fi
% ---------------------------------------------------------------------------
